
\documentclass{article}
%%%%%%%%%%%%%%%%%%%%%%%%%%%%%%%%%%%%%%%%%%%%%%%%%%%%%%%%%%%%%%%%%%%%%%%%%%%%%%%%%%%%%%%%%%%%%%%%%%%%%%%%%%%%%%%%%%%%%%%%%%%%%%%%%%%%%%%%%%%%%%%%%%%%%%%%%%%%%%%%%%%%%%%%%%%%%%%%%%%%%%%%%%%%%%%%%%%%%%%%%%%%%%%%%%%%%%%%%%%%%%%%%%%%%%%%%%%%%%%%%%%%%%%%%%%%
\usepackage{amsmath}

\setcounter{MaxMatrixCols}{10}
%TCIDATA{OutputFilter=LATEX.DLL}
%TCIDATA{Version=5.50.0.2960}
%TCIDATA{<META NAME="SaveForMode" CONTENT="1">}
%TCIDATA{BibliographyScheme=Manual}
%TCIDATA{Created=Wednesday, February 21, 2018 15:57:52}
%TCIDATA{LastRevised=Thursday, February 22, 2018 22:35:03}
%TCIDATA{<META NAME="GraphicsSave" CONTENT="32">}
%TCIDATA{<META NAME="DocumentShell" CONTENT="Standard LaTeX\Blank - Standard LaTeX Article">}
%TCIDATA{CSTFile=40 LaTeX article.cst}

\newtheorem{theorem}{Theorem}
\newtheorem{acknowledgement}[theorem]{Acknowledgement}
\newtheorem{algorithm}[theorem]{Algorithm}
\newtheorem{axiom}[theorem]{Axiom}
\newtheorem{case}[theorem]{Case}
\newtheorem{claim}[theorem]{Claim}
\newtheorem{conclusion}[theorem]{Conclusion}
\newtheorem{condition}[theorem]{Condition}
\newtheorem{conjecture}[theorem]{Conjecture}
\newtheorem{corollary}[theorem]{Corollary}
\newtheorem{criterion}[theorem]{Criterion}
\newtheorem{definition}[theorem]{Definition}
\newtheorem{example}[theorem]{Example}
\newtheorem{exercise}[theorem]{Exercise}
\newtheorem{lemma}[theorem]{Lemma}
\newtheorem{notation}[theorem]{Notation}
\newtheorem{problem}[theorem]{Problem}
\newtheorem{proposition}[theorem]{Proposition}
\newtheorem{remark}[theorem]{Remark}
\newtheorem{solution}[theorem]{Solution}
\newtheorem{summary}[theorem]{Summary}
\newenvironment{proof}[1][Proof]{\noindent\textbf{#1.} }{\ \rule{0.5em}{0.5em}}
\input{tcilatex}
\begin{document}


\part{Topics In Finance: Assignment 2}

\bigskip

\section{Introduction}

In this 2nd assignment, it was asked to estimate and plot the efficient
frontier of a portfolio of 49 industries, using the monthly returns for the
49 value weighted industry portfolio. Continually it was asked to calculate
the optimal Sharpe Ratio and finally calculate the allocations to each
industry portfolio which maximizes the Sharpe Ratio.

\bigskip

\section{Implementation}

For the estimation of the efficient frontiers, I use two approaches. Firstly
I estimate the efficient frontier, without allowing for short positions,
i.e. negative weitghts on assets, and in the 2nd approach, I relax this
constraint.

\subsection{Constrained Efficient Frontier and Sharpe Ratio}

Firstly, I estimate the efficient frontier for the 49 industries, without
allowing for short positions. The efficient frontier has been derived using
100 portfolios so as to better illustrate the line, because as the number of
portfolios increases, it means that we take into consideration more expected
return values, for the minimization of the risk (standard deviation). Of
course, after a sufficient number of portfolios, the shape of the efficient
frontier remains the same. As regards the results, the minimum variance
portfolio(MVP), has st. deviation of $\sigma =3.9345$ with expected return
of $E(r)~=0.99\%$ and the maximum return portfolio has st. deviation of \  $%
\sigma =9.2276$, achieving an expected return\ $E(r)~=1.43\%$.Considering
the optimal Sharpe ratio, this is the point of efficient frontier with
expected return $E(r)~=1.11\%$ and risk $\sigma =4.1709$.

In figure 1, the constrained efficient frontier is depicted. In figure 2,
the optimal shape ratio is depicted, i.e. the point of the efficient
frontier that maximizes the sharpe ratio. Finally, in figure 3, I
demonstrate the allocations to each industry portfolio which maximizies the
Sharpe Ratio.

\bigskip \qquad $\overset{\FRAME{itbpF}{5.8825in}{4.4157in}{0in}{}{}{Figure}{%
\special{language "Scientific Word";type "GRAPHIC";maintain-aspect-ratio
TRUE;display "USEDEF";valid_file "T";width 5.8825in;height 4.4157in;depth
0in;original-width 5.8228in;original-height 4.3647in;cropleft "0";croptop
"1";cropright "1";cropbottom "0";tempfilename
'P4KGEQ00.wmf';tempfile-properties "XPR";}}}{\
Fig.1:The~efficientfrontier,using~100~portfolios,for~the~49~value~weighted~industry%
}$\qquad \qquad \qquad \qquad \qquad \qquad

\qquad \qquad \qquad \qquad \qquad \qquad \qquad \qquad \qquad \qquad \qquad
\qquad \qquad \qquad \qquad \qquad \qquad \qquad

\bigskip \qquad $\overset{\FRAME{itbpF}{5.8825in}{4.4157in}{0in}{}{}{Figure}{%
\special{language "Scientific Word";type "GRAPHIC";maintain-aspect-ratio
TRUE;display "USEDEF";valid_file "T";width 5.8825in;height 4.4157in;depth
0in;original-width 5.8228in;original-height 4.3647in;cropleft "0";croptop
"1";cropright "1";cropbottom "0";tempfilename
'P4KGGR01.wmf';tempfile-properties "XPR";}}}{Fig.2:The\
optimal~SharpeRatio,for~the~constrained~portfolio}$\qquad \qquad \qquad
\qquad \qquad \qquad

\qquad \qquad \qquad \qquad \qquad \qquad \qquad \qquad \qquad \qquad \qquad
\qquad \qquad \qquad \qquad \qquad \qquad\ 

\ \ \ \ \ $\FRAME{itbpF}{5.8894in}{3.8614in}{0in}{}{}{Figure}{\special%
{language "Scientific Word";type "GRAPHIC";maintain-aspect-ratio
TRUE;display "USEDEF";valid_file "T";width 5.8894in;height 3.8614in;depth
0in;original-width 11.3057in;original-height 7.4028in;cropleft "0";croptop
"1";cropright "1";cropbottom "0";tempfilename
'P4KI2X02.bmp';tempfile-properties "XP";}}$\ \ \ \ \ \ \ \ \ \ \ \ \ \ \ \ \
\ \ \ \ \ \ \ \ \ \ \ \ \ \ \ \ \ \ \ \ \ \ \ \ \ \ \ \ \ \ \ \ \ \ \ \ \ \
\ \ \ \ \ \ \ 

$\overset{%
Fig.3:The~allocations~to~each~industry~portfolio~which~maximizes~the~sharperatio,~for~the~constrained~case%
}{}$

\subsection{Unconstrained Efficient Frontier and Sharpe Ratio}

For the case in which I allow for short positions, the MVP portfolio
generates an expected return of $E(r)=0.87\%$, with st. deviation $\sigma
=3.1442$, i.e. in comparison with the constrained case I achieve lower risk
with the cost of lower mean return. As regards the Maximum return portfolio,
this yields an expected return of  $E(r)=8.51\%$, i.e. a return much higher
than the constrained case, with the cost of higher risk $\sigma =63.5094$.
Considering the optimal Sharpe ratio, this is the point of efficient
frontier with expected return $E(r)=1.37$ and risk $\sigma =3.9384$. In
comparison with the optimal Sharpe ratio in the constrained case, allowing
for short positions leads in achieving higher expected return with lower
risk. Finally, as regards the allocations to each industry portfolio which
maximizies the Sharp

$\bigskip $

$\bigskip $

$\FRAME{itbpF}{5.8804in}{4.4145in}{0in}{}{}{Figure}{\special{language
"Scientific Word";type "GRAPHIC";maintain-aspect-ratio TRUE;display
"USEDEF";valid_file "T";width 5.8804in;height 4.4145in;depth
0in;original-width 5.8717in;original-height 4.4014in;cropleft "0";croptop
"1";cropright "1";cropbottom "0";tempfilename
'P4KJ3D06.wmf';tempfile-properties "XPR";}}$

$\overset{%
Fig.4:The~efficient~frontier,using~100~portfolios,for~the~49~value~weighted~industry,for~the~unconstrained~case.%
}{}$

$\FRAME{itbpF}{5.8804in}{4.4145in}{0in}{}{}{Figure}{\special{language
"Scientific Word";type "GRAPHIC";maintain-aspect-ratio TRUE;display
"USEDEF";valid_file "T";width 5.8804in;height 4.4145in;depth
0in;original-width 5.8717in;original-height 4.4014in;cropleft "0";croptop
"1";cropright "1";cropbottom "0";tempfilename
'P4KIKA04.wmf';tempfile-properties "XPR";}}$

$\overset{Fig.5:The\ optimal~Sharpe~Ratio,for~the~unconstrained~portfolio}{}$

\qquad \qquad \qquad \qquad \qquad \qquad \qquad \qquad \qquad \qquad \qquad
\qquad \qquad \qquad \qquad \qquad \qquad \qquad \qquad \qquad \qquad \qquad
\qquad \qquad \qquad \qquad \qquad \qquad \qquad \qquad \qquad \qquad\ 

$\FRAME{itbpF}{6.3408in}{4.1295in}{0in}{}{}{Figure}{\special{language
"Scientific Word";type "GRAPHIC";maintain-aspect-ratio TRUE;display
"USEDEF";valid_file "T";width 6.3408in;height 4.1295in;depth
0in;original-width 11.7364in;original-height 7.6333in;cropleft "0";croptop
"1";cropright "1";cropbottom "0";tempfilename
'P4KILD05.bmp';tempfile-properties "XPR";}}$

$\overset{%
Fig.6:The~allocations~to~each~industry~portfolio~which~maximizes~the~sharperatio,for~the~unconstrained~case.%
}{\ \ \ \ \ \ \ \ \ \ \ \ \ \ \ \ \ \ \ \ \ \ \ \ \ \ \ \ \ \ \ \ \ \ \ \ \
\ \ \ \ \ \ \ \ \ \ \ \ \ \ \ \ \ \ \ \ \ \ \ \ \ \ \ \ \ \ \ \ \ \ \ \ \ \
\ \ \ \ \ \ \ \ \ \ \ \ \ \ \ \ \ \ \ \ \ \ \ \ }$

\bigskip

\section{Conclusions}

In this assignment, I derived the efficient frontiers of a portfolio of 49
industries, using the monthly returns for the 49 value weighted industry
portfolio. Furthermore I estimated the optimal Sharpe Ratio and the
allocations to each industry portfolio which maximizes the Sharpe Ratio. For
these estimations, I used two methods, for constrained and unconstrained
portfolios, allowing in the 2nd case for short positions. I concluded that
allowing for short positions, I extend the lower risk and higher return
values of the efficient frontier and that the optimal Sharpe ratio point has
both higher return and lower risk than in the constrained case. Finally, as
regards the allocations of industries which maximize the sharpe ratio, I can
claim that in both cases, there are weights in all assets and that no asset
possess more than 12\% of the portfolio, resulting in a high diversified
portfolio.

\end{document}
