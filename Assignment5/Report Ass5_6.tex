
\documentclass[11pt]{article}
%%%%%%%%%%%%%%%%%%%%%%%%%%%%%%%%%%%%%%%%%%%%%%%%%%%%%%%%%%%%%%%%%%%%%%%%%%%%%%%%%%%%%%%%%%%%%%%%%%%%%%%%%%%%%%%%%%%%%%%%%%%%%%%%%%%%%%%%%%%%%%%%%%%%%%%%%%%%%%%%%%%%%%%%%%%%%%%%%%%%%%%%%%%%%%%%%%%%%%%%%%%%%%%%%%%%%%%%%%%%%%%%%%%%%%%%%%%%%%%%%%%%%%%%%%%%
\usepackage{amsfonts}
\usepackage{amsmath}

\setcounter{MaxMatrixCols}{10}
%TCIDATA{OutputFilter=LATEX.DLL}
%TCIDATA{Version=5.50.0.2953}
%TCIDATA{<META NAME="SaveForMode" CONTENT="1">}
%TCIDATA{BibliographyScheme=Manual}
%TCIDATA{Created=Thursday, March 15, 2018 13:02:50}
%TCIDATA{LastRevised=Thursday, March 15, 2018 13:19:43}
%TCIDATA{<META NAME="GraphicsSave" CONTENT="32">}
%TCIDATA{<META NAME="DocumentShell" CONTENT="Articles\SW\Mathematics Magazine Article">}
%TCIDATA{CSTFile=MathMag.cst}

\newtheorem{theorem}{Theorem}
\newtheorem{acknowledgement}[theorem]{Acknowledgement}
\newtheorem{algorithm}[theorem]{Algorithm}
\newtheorem{axiom}[theorem]{Axiom}
\newtheorem{case}[theorem]{Case}
\newtheorem{claim}[theorem]{Claim}
\newtheorem{conclusion}[theorem]{Conclusion}
\newtheorem{condition}[theorem]{Condition}
\newtheorem{conjecture}[theorem]{Conjecture}
\newtheorem{corollary}[theorem]{Corollary}
\newtheorem{criterion}[theorem]{Criterion}
\newtheorem{definition}[theorem]{Definition}
\newtheorem{example}[theorem]{Example}
\newtheorem{exercise}[theorem]{Exercise}
\newtheorem{lemma}[theorem]{Lemma}
\newtheorem{notation}[theorem]{Notation}
\newtheorem{problem}[theorem]{Problem}
\newtheorem{proposition}[theorem]{Proposition}
\newtheorem{remark}[theorem]{Remark}
\newtheorem{solution}[theorem]{Solution}
\newtheorem{summary}[theorem]{Summary}
\newenvironment{proof}[1][Proof]{\noindent\textit{#1.} }{}
\setlength{\topmargin}{-.3in}
\setlength{\textwidth}{5.5in}
\setlength{\textheight}{8.6in}
\setlength{\oddsidemargin}{0 in}
\renewcommand{\baselinestretch}{1.2}
\def\ds{\displaystyle}
\pagestyle{myheadings}
\parskip = 3mm
\markright{ Author }  
\thispagestyle{empty}

\input{tcilatex}

\begin{document}


\begin{center}
Assignment 5
\end{center}

\bigskip

\section{Introduction}

My task was to replicate for my momentum formation period criterion and
portfolio returns, similar tables to the ones discussed during class. I'll
proceed with the presentation of the three tables.

\bigskip

\section{Table 1}

\begin{equation*}
Table~1:~Summary~Statistics
\end{equation*}%
This table presents summary statistics for variables measuring momentum
using the Fama French Data library sample for the months from July 1926
through December 2017. Each month, the mean (Mean), standard deviation (SD),
skewness (Skew), excess kurtosis (Kurt), minimum (Min), fifth percentile
(5\%), 25th percentile (25\%), median (Median), 75th percentile (75\%), 95th
percentile (95\%), and maximum (Max) values of the cross-sectional
distribution of each variable are calculated. The table presents the
time-series means for each cross-sectional value.The column labeled n
indicates the average number of stocks per month, for which the given
variable is available.

\begin{description}
\item $%
\begin{tabular}{ccccccccccccc}
\hline
& $Mean$ & $SD$ & $Skew$ & $Kurt$ & $Min$ & $Min5\%$ & $25\%$ & $Median$ & $%
75\%$ & $95\%$ & $Max$ & $n$ \\ \hline
$Mom$ & $0.61$ & $6.46$ & $2.9$ & $49.0$ & $-76.04$ & $-10.2$ & $-4.22$ & $%
-6.86$ & $2.88$ & $10$ & $18.63$ & $46$%
\end{tabular}%
$
\end{description}

\bigskip

\bigskip

\section{Table 2}

\bigskip 
\begin{equation*}
Table~2:~Correlations
\end{equation*}%
This table presents the time-series averages of the monthy cross-sectional
Pearson product--moment (Panel A) and Spearman rank correlations (Panel B)
between momentum returns and of size and book to market (SMB).

\begin{equation*}
\begin{tabular}{l|l}
\hline
\multicolumn{2}{l}{Panel A: Pearson Cor.} \\ \hline
& Size \\ \cline{2-2}
Mom & $-0.24$ \\ \hline
\multicolumn{2}{l}{Panel B: Spearman Cor.} \\ \hline
& Size \\ \cline{2-2}
Mom & $-0.26$%
\end{tabular}%
\end{equation*}

\bigskip 

\section{Table 3}

\begin{equation*}
Table~3:Univariate~Portfolio~Analysis 
\end{equation*}

This table presents the results of univariate portfolio analyses of the
relation between each of the measures of momentum. Monthly portfolios are
formed by sorting all stocks in the sample into portfolios using decile
breakpoints calculated based on the given sort variable using all stocks of
the sample. Panel A shows the average values of Momentum in each decile
portfolio. Panel B (Panel C) shows the average equal-weighted
(value-weighted ) one-month-ahead excess return (in percent per month) for
each of the 10 decile portfolios. The table also shows the average return of
the portfolio that is long the 10th decile portfolio and short the first
decile portfolio (i.e my portfolio return using the momentum stratey), as
well as the CAPM and FF alpha for this portfolio. Newey and West (1987)
t-statistics, testing the null hypothesis that the average 10-1 portfolio
return or alpha is equal to zero, are shown in parentheses.%
\begin{equation*}
\begin{tabular}{cccccccccccccc}
\hline
& \multicolumn{13}{c}{Panel A: Mom-Sorted Portfolio Characteristics} \\ 
\hline
Value & $1$ & $2$ & $3$ & $4$ & $5$ & $6$ & $7$ & $8$ & $9$ & $10$ &  &  & 
\\ \hline
Mom & $0.90$ & $0.95$ & $0.97$ & $0.98$ & $1.00$ & $1.01$ & $1.02$ & $1.04$
& $1.06$ & $1.13$ &  &  &  \\ \hline
& \multicolumn{13}{c}{Panel B: Equal-Weighted Portfolio Returns} \\ \hline
Variable & $1$ & $2$ & $3$ & $4$ & $5$ & $6$ & $7$ & $8$ & $9$ & $10$ & $10-1
$ & $CAPM~a$ & $FF~a$ \\ \hline
Mom & $-0,20$ & $-0.09$ & $-0.03$ & $-0.01$ & $0.03$ & $0.05$ & $0.09$ & $%
0.12$ & $0.19$ & $0.32$ & $0.52$ & $0.52$ & $0.52$ \\ 
&  &  &  &  &  &  &  &  &  &  &  & $(25.87)$ & $(26.30)$ \\ \hline
& \multicolumn{13}{c}{Panel C: Value-Weighted Portfolio Returns} \\ \hline
Variable & $1$ & $2$ & $3$ & $4$ & $5$ & $6$ & $7$ & $8$ & $9$ & $10$ & $10-1
$ & $CAPM~a$ & $FF~a$ \\ \hline
Mom & $-0.4$ & $-0.17$ & $-0.06$ & $0.01$ & $0.06$ & $0.13$ & $0.21$ & $0.27$
& $0.4$ & $0.4$ & $0.8$ & $0.8$ & $0.81$ \\ 
&  &  &  &  &  &  &  &  &  &  &  & $(27.37)$ & $(27.71)$%
\end{tabular}%
\end{equation*}

\vspace*{0.2 in}

\end{document}
